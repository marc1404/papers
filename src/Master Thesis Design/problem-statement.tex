\section{Problem Statement}
The widespread adoption of e-commerce has established consumer generated reviews and ratings of products as a standard feature on merchant websites.
These reviews provide guidance for potential new customers and help to meet their expectations after receiving an ordered product.
Besides creating a knowledge database for consumers reviews contain valuable insights and measurements for both merchants and producers.
While not every single customer authors a review especially popular products amass numerous reviews on large marketplaces during their life cycle.
Reviews themselves differ in length, level of detail, quality of writing, structure, language and opinion features \cite{Hu2004a}.

However, e-commerce platforms of producers, merchants and large scale retailers are by no means the only place for users to express their opinion about products.
Review aggregation sites collect user reviews about products and services as a third party which does not offer or provide the reviewed subjects themselves.
Well established household names include: Metacritic\footnote{\url{http://www.metacritic.com/}} (entertainment), Rotten Tomatoes\footnote{\url{https://www.rottentomatoes.com/}} (movies, tv), IMDb\footnote{\url{http://www.imdb.com/}} (movies, tv, actors) and TripAdvisor\footnote{\url{https://www.tripadvisor.com/}} (vacation, flights, restaurants).
Aggregators act as independent platforms where users can find averaged reviews in a uniform fashion.

As a more specific instance with a much more narrow focus RateBeer\footnote{\url{https://www.ratebeer.com/}} collects information about (craft) beer and primarily consumer generated reviews of beers and breweries.
Established in 2000 it has since remained a popular exchange platform garnering more than five million reviews in total\footnote{\url{https://www.ratebeer.com/RateBeerBest/default_2013.asp}, Accessed: 16-03-2018} with a rate of roughly one million reviews per year since 2016\footnote{\url{https://www.ratebeer.com/ratebeerbest/default_2016.asp}, Accessed: 16-03-2018}.
While more recent and exact figures remain hidden it is estimated that nearly 500,000 visitors view the site within one month\footnote{\url{https://www.similarweb.com/website/ratebeer.com}, Accessed: 16-03-2018}.
Due to a number of factors RateBeer will be used as a testbed for this thesis project.
First, the large quantity of historic reviews and constant stream of new content provides an abundance of data for processing and experimental testing.
Second, according to RateBeer's quality assurance principles low quality and nonsense entries are swiftly removed from the platform through an extensive administration system.
Furthermore, although Anheuser-Busch InBev acquired a minority stake of RateBeer\footnote{\url{https://www.nytimes.com/2017/06/18/business/media/anheuser-busch-inbev-ratebeer.html}, Accessed: 16-03-2018} the site claims and emphasizes its mission to remain an independent platform which prohibits ratings by breweries and their affiliates.
Third, reviews are of a semi-structured nature containing free text but also scaled ratings in five distinct categories: (1) aroma, (2) appearance, (3) taste, (4) palate, and (5) overall.
Finally, being an international aggregator the site displays the location (city, country) of reviews when they were published by their authors.
This allows to perform analysis scoped to specific cities, regions or countries.

With an abundance of reviews at hand readers are easily overwhelmed by the sheer number of expressed opinions, and thus extracting meaningful information becomes a challenging task.
Existing solutions to this problem focus on different aspects in their summarization process.
In feature-based summaries \cite{Hu2004} product features are extracted and reviews are classified according to their sentiment towards these features.
This approach aims to reduce opinion bias when only a subset of reviews is read.
Similarly, this goal can also be achieved by ranking a set of questions about a product and matching reviews that are able to answer their respective question \cite{Liu2017}.
As other research has pointed out \cite{Liu2017, Al-Dhelaan2017} diversification is an important and sometimes overlooked characteristic of review summarization.
Besides including representative reviews into the summary diverse sentiments and aspects should be included as well.

Bearing this in mind, the summarization problem can be approached with a wider scope taking into account the dominance of social multimedia \cite{B2018a}.
The authors argue that summarization should be revisited in the light of modern multimedia to connect research areas which have been working in isolation from each other.
In order to meet today's information needs of users tapping single media sources is not sufficient anymore.
This allows to derive the following question which should lead this research project:

\noindent
\emph{How to generate representative and diverse product story summarizations driven by user generated reviews?}

As mentioned before RateBeer will act as a testbed for this experimental exploration.
It provides user generated reviews for a large quantity of beers (products) and enables the possibility to apply the story generation to breweries and regions as well.
The research will take other media sources into consideration such as further text or image content from social media or video advertisement which could be incorporated into the generated narrative.