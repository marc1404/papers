\section{Method}
In order to tackle the problem a system has to be devised which consists of four components building on each other.
Firstly, as the foundation for further work access to the review dataset on RateBeer has to be established.
This component collects historical reviews while also gathering new reviews being submitted to the site.
Secondly, summarization techniques will be applied to the acquired beer reviews.
Triage and evaluation of appropriate techniques will be supported by an initial literature study.
The summarization and story component is responsible for performing text analysis on the corpora and outputting story generation data.
Thirdly, the raw summarization and story output should be transformed in a visual representation which strikes a balance between factual and stylistic properties.
Finally, the last component is used to measure the performance and effectiveness of the system.

\subsection{Data Collection}
As RateBeer does not publish official datasets of its reviews alternative ways have to be utilized.
There are a few ways to approach this obstacle.

Using an open dataset\footnote{\url{https://s3.amazonaws.com/demo-datasets/beer_reviews.tar.gz}} with ~1,500,000 entries\footnote{\url{https://github.com/awesomedata/awesome-public-datasets/pull/339}, Accessed: 17-03-2018} can help to kick-start development of other system components while more reviews are collected in the meantime.

The official RateBeer API\footnote{\url{https://www.ratebeer.com/api-documentation.asp}, Accessed: 17-03-2018} is implemented in GraphQL\footnote{\url{https://graphql.org/}} but is not directly suitable for collecting reviews on a large scale.
It provides an endpoint for retrieving reviews of a beer given the respective ID number.
Consequently, given a list of beer IDs this endpoint could potentially be used to retrieve reviews in a reliable, structured way.
The default limitation of 5,000 requests per month and one request per second are an issue to keep in mind.
Overcoming the API limitations could be possible by contacting RateBeer's operators and discussing if data can be made available more easily with exclusively research purposes in mind.
In the past a dataset has been made available to researchers from Stanford University\footnote{\url{https://snap.stanford.edu/data/web-RateBeer.html}, Accessed: 17-03-2018} which indicates a chance that a formal request could be successful.

Provided that the previously described approaches are unable to generate a suitable dataset scraping can be used to collect review data.
This process involves programmatically visiting the website and extracting relevant data from the HTML response.
In a similar fashion to crawling bots of search engines links have to be discovered to new pages.
Essentially allowing the program to autonomously discover unknown beers and their respective reviews.
However, it has to be ensured that the scraping process does not produce high load on the platform or floods it with requests in a rogue manner.

\subsection{Summarization}
Due to the high number of reviews performing summarization tasks on the full dataset may become unfeasible on consumer devices.
As a support in this matter access has been granted to the Distributed ASCI Supercomputer (DAS) in its fourth generation \cite{Bal2016}.
Being designed for research areas with heavy demands for computational power and memory it is capable of handling the potentially large review dataset.
Furthermore, it is a system specifically built for running algorithms in parallel and on distributed nodes with an easy to use interface.

The selection and evaluation of appropriate natural language processing (NLP) techniques for summarization will be one of the primary tasks during the research.
Therefore, an initial literature study will be conducted to review currently used algorithms which can be deployed in this scenario.
One approach is to cluster semantically similar reviews into groups before identifying latent topics \cite{Nagwani2015}.
This second step is called topic modeling which reveals underlying key concepts in text.
Latent Dirichlet Allocation (LDA) \cite{Blei2003} is regularly, successfully \cite{Nagwani2015, Ren2013} used to generate probability topic models.
Notably, this approach can be combined with the MapReduce framework \cite{Nagwani2015} which is popular for processing large amounts of data.
Additionally, it is designed to be executed in parallel on distributed systems such as DAS-4.
TextRank \cite{Mihalcea2004} is a different algorithm that will be evaluated in this project.
It creates a graph-based ranking, inspired by web search page ranks, which can be used to extract keywords and summarize text.

\subsection{Evaluation}
Evaluating the performance and effectiveness of the proposed product summarization system is a challenging task.
There is no direct ground truth available to compare the generated summary artifacts to.
Nonetheless, official product descriptions by breweries and content from websites such as the \textit{Brewers Association}\footnote{\url{https://www.brewersassociation.org/}} or \textit{Brewers of Europe}\footnote{\url{http://brewersofeurope.org/}} could potentially be used as comparisons.
Popular scoring algorithms in the field of text summarization include ROUGE \cite{Lin2004} and the Pyramid Method \cite{Nenkova2004}.
They have been successfully employed in comparable research projects \cite{Nagwani2015, Ren2013}.