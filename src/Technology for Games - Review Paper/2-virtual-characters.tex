\section{Virtual Characters} \label{sec:virtual-characters}
Virtual worlds are usually filled with characters who are involved in the storyline and the gameplay itself.
\citeauthor{Schell2014} (2014) draws a comparison between fictional characters in novels, movies, and games.
While there certainly are similarities it shows that differences depend on the type of medium a character resides in.
Three patterns are identified for characters in games:
\begin{itemize}
    \item Since game characters rarely speak they are heavily involved in \textbf{physical} conflicts. In contrast to novels, the inner thoughts are usually not revealed which leaves thinking to the player.
    \item Instead of being build around reality game worlds tend to \textbf{fantasy} scenarios. This characteristic translates to inhabitants of these worlds as well.
    \item Albeit exceptions exist, a game's storyline and character depth are often \textbf{simpler} than more complex counterparts in movies and especially novels.
\end{itemize}
It is a fallacy to assume that games are bound to these patterns.
The implication is that it can require more effort to realize complex, meaningful, and coherent characters or stories in games.
Due to shrinking technical constraints, modern games are able to push these boundaries.

\subsection{Avatars}
The avatar is a special game character that is being controlled by the player \cite{Schell2014}.
Depending on the relationship between players and avatars there can be a large disconnection on one end of the spectrum or a strong mental projection on the other side.
In the latter case, the avatar becomes an extension of the player's identity.
Naturally, deciding between a first- and third-person perspective is difficult since it influences players ability to project themselves into their avatars.
There are two different archetypes for avatars which provide a basis for mental projection.
\begin{itemize}
    \item The \textbf{Ideal Form} describes a range of characters representing an idealized self of the player. They exhibit skills and attributes which they player does not have and is longing for.
    \item The \textbf{Blank Slate} uses an iconic approach where characters are easy to recognize but do not feature a lot of detail. Due to this lack, they provide opportunities for the player to connect with their avatar.
\end{itemize}
Finally, combining these types allows creating characters with attributes that players would like to have and at the same time are able to identify with.

\subsection{Agents}
Virtual agents, also referred to as virtual humans, are believable non-player characters (NPC) exhibiting human-like traits.
These intelligent agents are especially useful for training environments with a focus on human interaction.
The \emph{Virtual Humans} projects at the Institute for Creative Technologies (ICT) is such an example which aims to build high fidelity, embodied agents which can be used in simulated environments \cite{Kenny2007}.
Researchers at ICT identified three characteristics that should be conveyed for a successful implementation:
\begin{itemize}
    \item \textbf{Believability:} Agents have to achieve a minimum level of believability through the expression of human-like behavior in order to create an immersive experience for the actual human user.
    \item \textbf{Responsiveness:} They have to respond to both user actions and events in their environment.
    \item \textbf{Interpretability:} Users should be able to rely on known human cues to interpret the state of the virtual agent. This includes expressing the cognitive and emotional state through verbal and nonverbal responses.
\end{itemize}
However, realizing these characteristics in virtual agents also requires sophisticated graphics and animations as well as powerful computational power and capable artificial intelligence (AI).
As these virtual humans integrate various fields it is possible to devise a general architecture encompassing the three layers (inner to outer) \cite{Kenny2007}:
\begin{enumerate}
    \item \textbf{Cognitive Layer:} Contains the cognitive component, essentially the mind, of the agent. It outputs decisions depending on input, goals, and behavior. Depending on the specific scenario varying levels of cognitive ability are applicable for this layer.
    \item \textbf{Virtual Human Layer:} As the agent's body it is directly connected to the cognitive layer. It processes inputs such as vision and speech and produces outputs in the range of speech, gestures, and actions in the environment.
    \item \textbf{Simulation Layer:} This layer concerns itself with anything that is connected to the virtual environment.  It includes both software components necessary for the simulation but also hardware input from the real world. 
\end{enumerate} 
\citeauthor{Schell2014} (2014) enumerates useful advice for believable characters from which two are specifically applicable to virtual agents.

Firstly, he emphasizes the capability of human brains to process facial expression.
Furthermore, humans feature the most complex faces in the animal world.
It becomes apparent that facial expression is a crucial form of communication.
Therefore, incorporating facial expression in virtual characters is a powerful tool for conveying emotions.
Building on the ability of human brains to recognize these expressions they can be easily identified by users, even in the peripheral vision and without a lot of detail.

Secondly, virtual humans need to avoid the \emph{uncanny valley}.
The concept states that when characters reach a certain high level of human-likeness there is a region where they trigger negative, repulsive reactions.
This is often the case due to small mismatches in expectations when a subject seems like a human person but details are still off.
Designing characters with explicit human realism in mind carries a risk of entering the uncanny valley which should be accounted for.