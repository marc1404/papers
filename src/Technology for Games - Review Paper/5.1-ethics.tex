\subsection{Ethics}
\citeauthor{Sicart2005} (2005) discusses the importance and often overlooked aspect of games as ethical systems.
Existing ontologies describe games using all their structural properties including visuals, narratives, and interactions which constitute the experience.
However, the fictional level of games should be analyzed as well in order to work towards better understanding.
Digital games feature the special property of enforcing a set of rules
These rules are not subject to discussion and are encapsulated in a kind of black box that usually refrains from revealing its inner processes.
Naturally, virtual worlds are compared to the real world as a point of reference.
Rules constraining possibilities within games constitute an ethical system which is evaluated by players.
Reflecting on games as ethical objects allows to pose interesting questions in the range of ``To what extent does it matter what happens in fictional, virtual worlds?'', ``Are game designers responsible for making ethical restrictions?'', ``Is it beneficial if non-ethical actions are expressed in virtual worlds?''.