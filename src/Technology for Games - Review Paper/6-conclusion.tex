\section{Conclusion}
It becomes apparent when reviewing the literature selected in this paper that games do not exclusively serve their initial entertainment purpose.
Business, training scenarios, and behavioral modification practice all benefit from gamification and serious game.
Entertaining aspects of games, namely intrinsic motivation, is a powerful tool which can be applied to serious scenarios.
Featuring realistic virtual humans is a crucial component, especially for training purposes.
The boundaries of classic gaming are extended through adaptive gaming techniques dynamically generating content relative to the player profile. 
Virtual reality is arguably a field bearing a lot of potential for the mainstream entertainment industry.
Analyzed in the Gartner hype cycle\footnote{\url{https://www.gartner.com/it-glossary/hype-cycle}, accessed 06-04-2018} it currently travels the trough of disillusionment.
Technological advancements, specifically in GPUs, solved the most technical limitations.
However, ergonomics continue to be problematic leaving consumers disappointed after the excitement.

All in all, games continue to grow beyond being media for solely entertainment purposes.
The computational capabilities of modern computer systems allow game designers and researchers to focus on areas distinct from the gameplay itself.