\section{Introduction}
This paper provides an extensive overview of the current game literature which goes beyond games as merely entertainment objects.
Game worlds are inhabited by virtual characters which are either directly controlled by the player or the game itself (section \ref{sec:virtual-characters}).
Virtual agents are a vibrant research area which aims to create believable non-player characters (NPC).
Simulating realistic human-like behavior is especially useful for training purposes.
In these scenarios, it is often cheaper to avoid practice with real actors.
Serious gaming encompasses the field of applying entertaining games for the purpose of achieving a serious objective (section \ref{sec:serious-gaming}).
Cognitive training is a specific example which benefits from gamification to make repetitive practice tasks intrinsically motivating.
The gaming experience itself can be enhanced in both digital ways as well as building technology for better immersion (section \ref{sec:enhanced-gaming}).
Adaptive games rely on player models to dynamically adjust different aspects of the game.
Virtual reality and haptic feedback work towards creating a fully immersive experience that blocks out any signal from the real world.
Finally, games can be analyzed from a meta perspective using classifications, ontologies, and even ethical discussions (section \ref{sec:ontology}).
This allows describing games in a formal and precise way which creates a common level of understanding.