\section{Serious Gaming} \label{sec:serious-gaming}
Serious games combine a serious activity or goal with typical entertainment elements found in traditional games \cite{Boendermaker2015}.
Their objective is to increase and sustain motivation for players to keep participating in the serious component of the game.
The motivational factors used in games can be categorized according to the two types of motivation: intrinsic or extrinsic.
Intrinsic motivation stems from doing an inherently enjoyable activity (e.g. playing a game).
Extrinsic motivation is fueled by having a desirable but separable outcome to an activity (e.g. additional in-game currency) \cite{Ryan2000}.

\subsection{Intrinsic Motivation}
Intrinsic motivators are concerned with retaining the interest of the player \cite{Birk2016}.
This can be achieved in four different ways reaching players on varying psychological levels: (1) enjoyment, (2) competence, (3) autonomy, and (4) relatedness.

(1) \textbf{Enjoyment} suggests adding novelty through new content.
This boils down to additional levels, characters, items, objectives, and story elements.

(2) \textbf{Competence} is a psychological need of players that can be satisfied by overcoming obstacles and challenges.
While players improve their skills in games they are rewarded with a feeling of accomplishment which is usually confirmed and reinforced through game feedback.
Serious games, such as training objectives, can suffer from a steep learning curve for acquiring skills.
This negatively affects players as it leaves them with a fading satisfaction of competence.
In order to counteract this effect games should introduce additional skills or aim to lower steep learning curves.

(3) \textbf{Autonomy} is achieved when players are able to make their own decisions with a feeling of freedom.
Games can nudge players towards making certain decisions, the illusion of freedom should be sustained.
Choices are commonly provided through character customization but also branching narratives.
Serious games should have a special focus on ensuring that players autonomously progress into more difficult training activities.

(4) \textbf{Relatedness} occurs when players create social connections with other players or in-game characters.
Multiplayer games inherently satisfy this need using team cooperation or friend networks.
However, singleplayer games are able to rely on virtual characters with whom the player can form relationships.

\subsection{Extrinsic Motivation}
Extrinsic motivation provides separate, desirable outcomes to the player which can be realized using explicit rewards \cite{Birk2016}.
However, psychological principles apply as well.
Four approaches increasing extrinsic motivation are identified: (1) external regulation, (2) introjection, (3) identification, and (4) integration.

(1) \textbf{External regulation} through the means of rewards is a common approach.
Rewards are beneficial for players in a broader scope but they are separate from the objectives players are pursuing.

(2) \textbf{Introjection} is achieved when players receive approval from others or even themselves.
This is implemented using rare status items or hard to earn ranks and achievements.
These introjection elements may not have a direct use in the game but they carry prestigious value creating a feeling of importance.

(3) \textbf{Identification} reinforces self-endorsement of activities with a greater goal.
Seemingly dull and repetitive tasks are accepted when players see the value in completing these tasks.

(4) \textbf{Integration} describes motivation created by aligning the goals of players with their self-view.
Internal beliefs are the driving factor for taking on challenges.

\subsection{Cognitive Bias Modification}
Disrupting addictive behavioral patterns, specifically in regard to substance abuse, is a major challenge in society \cite{Boendermaker2015}.
It has been shown that intervening during the formative period of adolescence is an effective measure to prevent later negative developments.
Interventions fall into one of two categories by being explicit or implicit.
Explicit interventions can utilize clear warning messages or personalized interview which directly touch upon the topic in question.
The effectiveness of explicit techniques has been criticized pushing towards implicit approaches such as cognitive training.
Addictive behavior is often triggered due to strong impulsive reactions overriding relatively weak control processes.
Cognitive Bias Modification (CBM) aims to reduce the imbalance by providing more time for decision making and strengthening the control processes in general.
In practice, CBM is already realized using computer-based reaction time tasks.
As training involves extensive, repetitive practice sessions adding game elements is a viable approach for making the technique more entertaining.
Selective attention is a typical training area for CBM which relies on Visual Probe Tasks (VPT).
Six steps work towards gamification of this implicit, cognitive technique \cite{Boendermaker2015}:
\begin{enumerate}
    \item As a first step, simple game elements are added to the evidence-based task. They are of extrinsic motivational nature and thus usually reward systems. Audible and visible feedback responses are another way of encouraging players to progress.
    \item Intrinsic motivation is the next crucial step in order to make the training task itself more enjoyable. External rewards may be able to keep participants in the loop but transforming the training task into an entertaining game proves to be more effective.
    \item Instead of applying intrinsic motivational features to an existing training task it is possible to start with a game that adheres to the principle of an evidence-based task. However, it might be more difficult to identify which bias modification contributed to the desired effect. While this does not prevent the training game from being effective it complicates research.
    \item Adding a game-shell around the training task is a further, progressive approach. The experience resembles a typical game structure while the original training task remains unmodified. Actual virtual environments are an advantage of this approach as they do not alter the training task but support the gamification goal.
    \item This step combines the intrinsic integration approach with the game-shell. As the game-shell alone can be interpreted as an extrinsic motivator adding intrinsic motivators leads to a better result. Nonetheless, this combination is arguably difficult to implement as it is a threefold integration of CBM ideas with intrinsic motivators as well as the extrinsic reward system of the game-shell.
    \item Finally, off-the-shelf (OTS) games may be used as the starting point for improving selective attention. Games stemming from the entertainment industry naturally fulfill the motivational aspect. However, integrating the CBM specific topics into existing OTS games proves to be difficult or even unfeasible for a range of games.
\end{enumerate}