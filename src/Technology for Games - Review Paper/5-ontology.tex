\section{Ontology} \label{sec:ontology}
\citeauthor{Elverdam2007} (2007) argue that the two common ways of describing games, namely by direct comparison or genre categorization, are not sufficient for a precise classification.
Direct comparison usually relies on one specific, similar connection but does not account for all the other differences.
Assigning games into genres works for a broader categorization but it also falls short of identifying inter-genre differences.
The authors iterate on top of the open-ended typology by \citeauthor{Aarseth2003} (2003) which contains sixteen dimensions grouped into six meta-categories.
They propose a modification and extend the meta-categories to eight in total: (1) virtual space, (2) physical space, (3) internal time, (4) external time, (5) player composition, (6) player relation, (7) struggle, and (8) game state.
Each meta-category features deeper dimensions for further classification.

\textbf{Virtual Space:}
\begin{itemize}
    \item \textbf{Perspective:} In an omnipresent perspective the player is able to see a total overview while a vagrant perspective requires strategic movement.
    \item \textbf{Positioning:} Absolute positioning allows to precisely discern the player's position. Alternatively, relative positioning creates a dependency between the player's position and other objects.
    \item \textbf{Environment:} Three different ways for the player to make modifications to the game space: (1) free, (2) fixed, or (3) none.
\end{itemize}

\textbf{Physical Space:}
\begin{itemize}
    \item \textbf{Positioning:} Omnipresent allows viewing the entire physical game space while vagrant requires movement.
    \item \textbf{Positioning:} The player 's position can be relatively determined in relation to other game agents (proximity based) or both factors combined.
\end{itemize}

\textbf{External Time:}
\begin{itemize}
    \item \textbf{Teleology:} Games can end after a finite amount of time or carry on infinitely.
    \item \textbf{Representation:} Whether time mimics reality or behaves arbitrarily.
\end{itemize}

\textbf{Internal Time:}
\begin{itemize}
    \item \textbf{Haste:} Present haste creates a connection between the game state and real-time. Absent haste indicates no such connection.
    \item \textbf{Synchronicity:} If it is present game agents are able to act at the same time.
    \item \textbf{Interval Control:} With present control, players can trigger the next game cycle on their own.
\end{itemize}

\textbf{Player Composition:}
\begin{itemize}
    \item Different types of player organization: single player, single team, two players, two teams, multiplayer, or multiteam.
\end{itemize}

\textbf{Player Relation:}
\begin{itemize}
    \item \textbf{Bond:} Player relations can change dynamically or are static.
    \item \textbf{Evaluation:} Different forms of outcome quantification: individual players, as teams, or both.
\end{itemize}

\textbf{Struggle:}
\begin{itemize}
    \item \textbf{Challenge:} Three ways of opposition in games: (1) identical are predefined challenges which are always the same, (2) instance uses a predefined framework which varies through randomness, or (3) agents which act autonomously.
    \item \textbf{Goals:} Games can have absolute goals which remain fixed or relative towards the unique events in a game session.
\end{itemize}

\textbf{Game State:}
\begin{itemize}
    \item \textbf{Mutability:} Changes in the game state can be temporal, finite for a game session, or infinite by spanning multiple game sessions.
    \item \textbf{Savability:} The conditions for the player to save and restore the game state being either unlimited, conditional or no savability.
\end{itemize}

The Game Ontology Project (GOP) as presented by \citeauthor{Zagal2007} (2007) is a fully fledged framework for describing, analyzing, and reasoning about games.
It is a more detailed approach in comparison to the classification model and aims to identify structural game elements as well as their relationships in a hierarchical organization.
Game design spaces are characterized by focusing on components that cause, effect, and relate to actual gameplay.
The GOP uses concepts from prototype theory in combination with a methodological, inductive approach from grounded theory to generate the ontology.
In order to characterize game design spaces the representational specifics of games are represented in an abstract way in this framework.
Five top-level elements constitute the ontology: (1) interface, (2) rules, (3) goals, (4) entities, and (5) entity manipulation.

\begin{itemize}
    \item \textbf{Interface:} This is the bi-directional communication area between the player and the game. It allows players to take in-game actions by actuating input devices of various forms. Physical signals are mapped to digital actions in the game. In the reverse direction, the game provides sensory feedback to the player usually in the form of visual graphics and aural sensation.
    \item \textbf{Rules:} Define the constraints and level of freedom in a game. They describe how a game unfolds and the possible interactions within. Rules can be further differentiated into gameplay and game world rules. The latter affects the whole game world and are commonly used in simulator type games. Gameplay rules lay down the specifics of game flow such as the abilities and meta attributes of players.
    \item \textbf{Goals:} Goals encompass in-game objectives that must be accomplished in order to win. While they are clearly defined they are not necessarily communicated to the player. Goals can be analyzed at different levels as there are overarching objectives for games which contain subordinate goals which are required for their completion.
    \item \textbf{Entities:} The objects in the reality of the game world. It is emphasized by the authors that this section requires future development. This is justified by the fact that entities are implicitly defined by entity manipulations.
    \item \textbf{Entity Manipulation:} Entities have attributes and abilities which are altered by entity manipulations. In this sense, abilities are the verbs of entities which are the nouns of the game world. Subsequently, attributes are the adjectives being changed by abilities. Entities lacking abilities are categorized as being static as they do not manipulate their environment. In contrast, dynamic entities use their abilities to change attributes of other entities.
\end{itemize}

\subsection{Ethics}
\citeauthor{Sicart2005} (2005) discusses the importance and often overlooked aspect of games as ethical systems.
Existing ontologies describe games using all their structural properties including visuals, narratives, and interactions which constitute the experience.
However, the fictional level of games should be analyzed as well in order to work towards better understanding.
Digital games feature the special property of enforcing a set of rules
These rules are not subject to discussion and are encapsulated in a kind of black box that usually refrains from revealing its inner processes.
Naturally, virtual worlds are compared to the real world as a point of reference.
Rules constraining possibilities within games constitute an ethical system which is evaluated by players.
Reflecting on games as ethical objects allows to pose interesting questions in the range of ``To what extent does it matter what happens in fictional, virtual worlds?'', ``Are game designers responsible for making ethical restrictions?'', ``Is it beneficial if non-ethical actions are expressed in virtual worlds?''.