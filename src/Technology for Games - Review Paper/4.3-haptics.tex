\subsection{Haptics} \label{sec:haptics}
While the world continues to being transformed by digitalization a lacking area is haptics of virtual objects.
Tactile technology has progressed with large companies such as Apple embedding the \emph{taptic engine} in all of its recent devices in order to provide tactile feedback under the marketing name \emph{force touch}.
However, the majority of feedback devices require an interactive surface or wearable haptic equipment.

AIREAL is a novelty tactile feedback system utilizing precise air vortices to create expressive sensations \cite{Sodhi2013}.
In combination with interactive displays or virtual reality (see also section x.x) it bears the potential of allowing users to feel 3D virtual objects without touching or wearing additional haptic equipment.
This fosters blurring the line between the virtual and real world.

Air vortices are created through pressure differences in a relatively small cube with a nozzle that can be freely actuated.
Their advantages are a relatively long travel distance (up to one meter), efficiency, low costs, and scalability.

Five principles steered the design of applications for the AERIAL haptics system \cite{Sodhi2013}:
\begin{itemize}
    \item \textbf{Collocation:} Visual images and tactile sensations should be collocated in space and time in the sense of an overlap between projected images and haptic feedback on the user's body.
    \item \textbf{Persistence:} Static areas and objects are able to emit fixed tactile simulations representing real physical objects.
    \item \textbf{Variance:} The haptic sensations can represent varying, rich textures in 3D space.
    \item \textbf{Continuity:} The tactile feedback can be moved continuously around the user.
    \item \textbf{Transience:} Air haptics are able to actuate real physical objects in the environment near the user.
\end{itemize}

The authors identify two main limitations of the system \cite{Sodhi2013}.
Firstly, an audible knock is produced due to the use of a high amplitude and low-frequency signal.
Secondly, while it achieves its goal of avoiding active instrumentation for the user it still requires passive instrumentation for operations in the user's environment.