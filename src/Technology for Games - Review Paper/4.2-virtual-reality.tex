\subsection{Virtual Reality} \label{sec:virtual-reality}
Virtual reality (VR) has garnered a lot of attention and hype in recent years.
It is regarded as a novelty which naturally spikes interest in people.
However, technical limitations still prevent mainstream adoption.
This leaves consumers disappointed and somewhat disillusioned about the hype in VR.
Interestingly enough the concept of VR has been established for quite some time ago with Ivan Sutherland's vision in 1965 \cite{Brooks1999}.
Working implementations have been published in 1994 with the first production systems in 1999 already.

Four technologies are crucial components for VR systems and essentially encompass the definition \cite{Brooks1999}:
\begin{itemize}
    \item Visual, aural, and haptics displays immersing the user in a virtual world. Contradicting sensory signals from the real world are blocked out.
    \item The graphics rendering systems constantly streaming images with at least thirty frames per second to simulate a fluid reality.
    \item The tracking system providing information about the position and orientation of the user's head and extremities.
    \item The system responsible for building and maintaining high fidelity of the virtual world to create a realistic experience.
\end{itemize}
While these components provide the basics for VR additional technologies are able to improve the immersive feeling including synthetic directional sound and sound fields, tactile and kinesthetic feedback (see also section \ref{sec:haptics}), specialized tracking gloves enabling better interactions with the virtual world, and finding adequate replacement interactions for their counterparts in the real world.

Multiple areas proved to be challenging for the usage of VR in production environments: rendering engines, tracking, ergonomics, and latency \cite{Brooks1999}.
The most problematic issues are solved to an acceptable degree which allowed the first consumers to get in touch with VR systems.
However, specific issues in rendering engines continue to exist and especially the ergonomics of equipment requires further improvement.
Exponentially growing power of graphics processing units (GPU) in recent years solved the most pressing matter of rendering engines.
Head-strapped displays lacking retina resolution carry the inherent problem that users are able to see the underlying grid due to the eyes being in close proximity to the display surface.
This issue will certainly be rendered obsolete with the fast-paced progression of display technology.
Ergonomics are still problematic since powerful VR systems require a wire from the worn equipment to the computational unit.
Wireless systems certainly exist but lack far behind the capability of dedicated desktop stations.