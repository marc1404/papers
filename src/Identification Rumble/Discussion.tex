\section{Discussion}

This section provides highlights from the user testing feedback, several limitations to the testing are discussed, recommendations are given on different implementation strategies and both required and optional future work is listed. 

\subsection{User Feedback}
As discussed, user feedback has had a substantial impact on refinements of the prototype. The most important positive feedback includes the positive connection between museum visitors, preferences for animation over text, the replica of the identity card, and the general engagement of the visitors with the dilemmas. 

Most of the received feedback described in section \ref{US_TEST} has been implemented in next iterations of the prototype. Some important feedback that has not been fully implemented in the current prototype includes more encouragement for groups to discuss the answer among themselves, the balance between the text being too fast and the animation being too long, and the visitor's choices have no real world consequences for them. 

\subsection{Limitations}
The presented user tests and methodology include several possible biases. Most importantly, results could be skewed because during user tests, researchers were present to guide test participants through the experience and were asked questions. In its finalized form, this would not be possible. 
Secondly, too few user tests were carried out to provide statistically significant results, though results were highly useful for feedback and ideas for improvements. The implementation of Identification Rumble would require the volunteers of the Dutch Resistance Museum to keep track of additional physical objects, which could be a hindrance for them. User tests in the museum were carried out in the main hall, just in front of the start of the exhibits. This might have influenced user experiences. Also, the prototype is not complete with historically accurate imagery in the animation. The voice over and discussion prompting texts that were added in the final prototype were not subjected to user texts and the final prototype has minor differences in the lay out compared to the lay out the exhibits in their implemented form. 
Finally, the museum could adopt the idea to gift the visitors the identity cards as their production cost is quite low and it might function as a distinct souvenir to take home and remember the museum by. 

\subsection{Conclusion and Recommendation}
Identification Rumble first and foremost provides an interactive experience for visitors. The combination of an animation and the possibility to answer for visitors to give their own answers is engages the visitors to think about the dilemma. This shows a stark contrast to the current exhibits where some visitors do not even notice some dilemmas. Evidently, the implementation of Identification Rumble is recommended. However, this can be done do different extents. 

Firstly, the dilemma does not have to be implemented for every dilemma to make a difference. Current exhibits could gradually be replaced by dilemma stations equipped with RFID readers. However, it is recommended that first implementation should cover at least 5 dilemmas. Otherwise, the role of the evaluation would be far less significant. 

Moreover, the dilemmas do not have to be presented in the same format. For example, dilemmas could differ from each other in terms of duration, interactivity and technology. Also, not all dilemmas necessarily need an animation, though test users did favor animation over plain text. 
Lastly, dilemma stations could be implemented without the RFID system and identity cards entirely and use buttons instead. This does complicate choosing language and eliminates the option for a personal evaluation at the end of the tour. 

\subsection{Future Work}
In order for Identification Rumble to be implemented, some work is still needed. 
To finalize the dilemma \textit{'Register?'}, the animation needs text and voice overs in all the languages used by the Dutch Resistance Museum. 
Also, professional voice overs should be recorded and all copyrighted imagery in the animation must be replaced. 
Furthermore, to extend Identification Rumble to multiple dilemmas, animations are needed as well as historically accurate stories and voice acting in multiple languages. 
In terms of hardware, every dilemma station would also require two RFID readers. 
Then, the language station would require a RFID scanner for every available language, currently six. 
Finally, the evaluation station could reveal personalized stories based on the choices that were made with a single identity card. 

%Fonds21\footnote{\url{https://www.fonds21.nl/}, Accessed: 08-2-2018} and VSBfonds\footnote{\url{https://www.vsbfonds.nl/}, Accessed: 08-2-2018} can be a solution for the money problem.