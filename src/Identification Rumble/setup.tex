\documentclass[sigconf]{acmart}

\usepackage{booktabs}
\usepackage{hyperref}
\usepackage{listings}
% * <marc.vornetran@gmail.com> 2018-03-02T13:58:29.061Z:
%
% ^.
\usepackage{color}

\newcommand\todo[1]{\textbf{\textcolor{red}{#1}}}

\input{javascript-listing}

\settopmatter{printacmref=false}
\renewcommand\footnotetextcopyrightpermission[1]{} % removes footnote with conference information in first column
\pagestyle{plain} % removes running headers

\begin{document}
\title{Identification Rumble}
\subtitle{Increasing interaction between visitors and World War II dilemmas in the Dutch Resistance Museum}

\author{Maaike Koolbergen}
\affiliation{
  \institution{University of Amsterdam}
  \city{Amsterdam}
  \country{The Netherlands}
}
\email{maaikekoolbergen@gmail.com}

\author{Gokie Wiegers}
\affiliation{
  \institution{University of Amsterdam}
  \city{Amsterdam}
  \country{The Netherlands}
}
\email{gokie.wiegers@gmail.com}

\author{Jerom Fernig}
\affiliation{
  \institution{University of Amsterdam}
  \city{Amsterdam}
  \country{The Netherlands}
}
\email{jeromfernig@gmail.com}

\author{Marc Vornetran}
\affiliation{
  \institution{University of Amsterdam}
  \city{Amsterdam}
  \country{The Netherlands}
}
\email{marc.vornetran@gmail.com}


\begin{abstract}
The main exhibition of the Dutch Resistance Museum in Amsterdam about the Netherlands in World War II illustrates personal experiences of Dutch citizens during the occupation but has been largely left unchanged for almost two decades. The museum has asked Master of Science students from the University of Amsterdam to develop one or more interactive museum exhibits that present twelve main dilemmas Dutch citizens faced during the occupation of the Netherlands by Nazi Germany. Through user-centered design and extensive iteration, an interactive and engaging experience is proposed. This paper presents the design, development and use of the interactive museum experience: \textit{Identification Rumble}. 
\end{abstract}

\keywords{Tangible interactions, interaction design, user experience, smart replica, interactive museum exhibit, Dutch Resistance Museum}

\maketitle
\section{Introduction}
The widespread adoption of e-commerce has established consumer-generated reviews and ratings of products as a standard feature on merchant websites.
These reviews provide guidance for potential new customers and help to meet their expectations after receiving an ordered product.
Besides creating a knowledge database for consumers reviews contain valuable insights and measurements for both merchants and producers.
While not every single customer authors a review especially popular products amass numerous reviews on large marketplaces during their life cycle.
Reviews themselves differ in length, level of detail, quality of writing, structure, language, and opinion features \cite{Hu2004a}.

However, e-commerce platforms of producers, merchants and large-scale retailers are by no means the only place for users to express their opinion about products.
Review aggregation sites collect user reviews about products and services as a third party which does not offer the reviewed subjects themselves.
Well established household names include: \textit{Metacritic}\footnote{\url{http://www.metacritic.com}} (entertainment), \textit{Rotten Tomatoes}\footnote{\url{https://www.rottentomatoes.com}} (movies, tv), \textit{IMDb}\footnote{\url{http://www.imdb.com}} (movies, tv, actors) and \textit{TripAdvisor}\footnote{\url{https://www.tripadvisor.com}} (vacation, flights, restaurants).
Aggregators act as independent platforms where users can find averaged reviews in a uniform fashion.

As a more specific instance with a much more narrow focus \textit{RateBeer}\footnote{\url{https://www.ratebeer.com}} collects information about (craft) beer and primarily consumer-generated reviews of beers and breweries.
Established in 2000 it has since remained a popular exchange platform garnering more than five million reviews in total\footnote{\url{https://www.ratebeer.com/RateBeerBest/default_2013.asp}, accessed: 16-03-2018} with a rate of roughly 1 million reviews per year since 2016\footnote{\url{https://www.ratebeer.com/ratebeerbest/default_2016.asp}, accessed: 16-03-2018}.
While more recent and exact figures remain hidden it is estimated that nearly 500,000 visitors view the site within a single month\footnote{\url{https://www.similarweb.com/website/ratebeer.com}, accessed: 16-03-2018}.

Due to the following factors, \textit{RateBeer} acts as a testbed for this thesis project.
Firstly, the large quantity of historical reviews and the constant stream of new content provides an abundance of data for processing and experimental testing.
Secondly, according to \textit{RateBeer's} quality assurance principles, low quality and nonsensical entries are swiftly removed from the platform through an extensive administration system.
Furthermore, although \textit{Anheuser-Busch InBev} acquired a minority stake of \textit{RateBeer}\footnote{\url{https://www.nytimes.com/2017/06/18/business/media/anheuser-busch-inbev-ratebeer.html}, accessed: 16-03-2018} the site claims and emphasizes its mission to remain an independent platform which prohibits ratings by breweries and their affiliates.
Finally, reviews are of a semi-structured nature containing free-form text but also scaled ratings in 5 distinct categories: (1) aroma, (2) appearance, (3) taste, (4) palate, and (5) overall.

The following parts of this paper are structured as follows.
Firstly, section \ref{sec:related-work} examines related work in the fields of word embeddings and (visual) summarization.
Secondly, section \ref{sec:problem-statement} contains the problem statement alongside the pursued research questions.
Thirdly, section \ref{sec:method} details the approach taken for the realization of the \textit{Beerlytics} system going from data collection and analysis, over the system design of the API and frontend prototype, to the final evaluation.
Fourthly, section \ref{sec:discussion} discusses actual and potential shortcomings of the research project.
Finally, section \ref{sec:conclusion-future-work} concludes this research by recapping the proposed system and laying out recommendations for future work.
\section{Virtual Characters} \label{sec:virtual-characters}
Virtual worlds are usually filled with characters who are involved in the storyline and the gameplay itself.
\citeauthor{Schell2014} (2014) draws a comparison between fictional characters in novels, movies, and games.
While there certainly are similarities it shows that differences depend on the type of medium a character resides in.
Three patterns are identified for characters in games:
\begin{itemize}
    \item Since game characters rarely speak they are heavily involved in \textbf{physical} conflicts. In contrast to novels, the inner thoughts are usually not revealed which leaves thinking to the player.
    \item Instead of being build around reality game worlds tend to \textbf{fantasy} scenarios. This characteristic translates to inhabitants of these worlds as well.
    \item Albeit exceptions exist, a game's storyline and character depth are often \textbf{simpler} than more complex counterparts in movies and especially novels.
\end{itemize}
It is a fallacy to assume that games are bound to these patterns.
The implication is that it can require more effort to realize complex, meaningful, and coherent characters or stories in games.
Due to shrinking technical constraints, modern games are able to push these boundaries.

\subsection{Avatars}
The avatar is a special game character that is being controlled by the player \cite{Schell2014}.
Depending on the relationship between players and avatars there can be a large disconnection on one end of the spectrum or a strong mental projection on the other side.
In the latter case, the avatar becomes an extension of the player's identity.
Naturally, deciding between a first- and third-person perspective is difficult since it influences players ability to project themselves into their avatars.
There are two different archetypes for avatars which provide a basis for mental projection.
\begin{itemize}
    \item The \textbf{Ideal Form} describes a range of characters representing an idealized self of the player. They exhibit skills and attributes which they player does not have and is longing for.
    \item The \textbf{Blank Slate} uses an iconic approach where characters are easy to recognize but do not feature a lot of detail. Due to this lack, they provide opportunities for the player to connect with their avatar.
\end{itemize}
Finally, combining these types allows creating characters with attributes that players would like to have and at the same time are able to identify with.

\subsection{Agents}
Virtual agents, also referred to as virtual humans, are believable non-player characters (NPC) exhibiting human-like traits.
These intelligent agents are especially useful for training environments with a focus on human interaction.
The \emph{Virtual Humans} projects at the Institute for Creative Technologies (ICT) is such an example which aims to build high fidelity, embodied agents which can be used in simulated environments \cite{Kenny2007}.
Researchers at ICT identified three characteristics that should be conveyed for a successful implementation:
\begin{itemize}
    \item \textbf{Believability:} Agents have to achieve a minimum level of believability through the expression of human-like behavior in order to create an immersive experience for the actual human user.
    \item \textbf{Responsiveness:} They have to respond to both user actions and events in their environment.
    \item \textbf{Interpretability:} Users should be able to rely on known human cues to interpret the state of the virtual agent. This includes expressing the cognitive and emotional state through verbal and nonverbal responses.
\end{itemize}
However, realizing these characteristics in virtual agents also requires sophisticated graphics and animations as well as powerful computational power and capable artificial intelligence (AI).
As these virtual humans integrate various fields it is possible to devise a general architecture encompassing the three layers (inner to outer) \cite{Kenny2007}:
\begin{enumerate}
    \item \textbf{Cognitive Layer:} Contains the cognitive component, essentially the mind, of the agent. It outputs decisions depending on input, goals, and behavior. Depending on the specific scenario varying levels of cognitive ability are applicable for this layer.
    \item \textbf{Virtual Human Layer:} As the agent's body it is directly connected to the cognitive layer. It processes inputs such as vision and speech and produces outputs in the range of speech, gestures, and actions in the environment.
    \item \textbf{Simulation Layer:} This layer concerns itself with anything that is connected to the virtual environment.  It includes both software components necessary for the simulation but also hardware input from the real world. 
\end{enumerate} 
\citeauthor{Schell2014} (2014) enumerates useful advice for believable characters from which two are specifically applicable to virtual agents.

Firstly, he emphasizes the capability of human brains to process facial expression.
Furthermore, humans feature the most complex faces in the animal world.
It becomes apparent that facial expression is a crucial form of communication.
Therefore, incorporating facial expression in virtual characters is a powerful tool for conveying emotions.
Building on the ability of human brains to recognize these expressions they can be easily identified by users, even in the peripheral vision and without a lot of detail.

Secondly, virtual humans need to avoid the \emph{uncanny valley}.
The concept states that when characters reach a certain high level of human-likeness there is a region where they trigger negative, repulsive reactions.
This is often the case due to small mismatches in expectations when a subject seems like a human person but details are still off.
Designing characters with explicit human realism in mind carries a risk of entering the uncanny valley which should be accounted for.
\section{Serious Gaming} \label{sec:serious-gaming}
Serious games combine a serious activity or goal with typical entertainment elements found in traditional games \cite{Boendermaker2015}.
Their objective is to increase and sustain motivation for players to keep participating in the serious component of the game.
The motivational factors used in games can be categorized according to the two types of motivation: intrinsic or extrinsic.
Intrinsic motivation stems from doing an inherently enjoyable activity (e.g. playing a game).
Extrinsic motivation is fueled by having a desirable but separable outcome to an activity (e.g. additional in-game currency) \cite{Ryan2000}.

\subsection{Intrinsic Motivation}
Intrinsic motivators are concerned with retaining the interest of the player \cite{Birk2016}.
This can be achieved in four different ways reaching players on varying psychological levels: (1) enjoyment, (2) competence, (3) autonomy, and (4) relatedness.

(1) \textbf{Enjoyment} suggests adding novelty through new content.
This boils down to additional levels, characters, items, objectives, and story elements.

(2) \textbf{Competence} is a psychological need of players that can be satisfied by overcoming obstacles and challenges.
While players improve their skills in games they are rewarded with a feeling of accomplishment which is usually confirmed and reinforced through game feedback.
Serious games, such as training objectives, can suffer from a steep learning curve for acquiring skills.
This negatively affects players as it leaves them with a fading satisfaction of competence.
In order to counteract this effect games should introduce additional skills or aim to lower steep learning curves.

(3) \textbf{Autonomy} is achieved when players are able to make their own decisions with a feeling of freedom.
Games can nudge players towards making certain decisions, the illusion of freedom should be sustained.
Choices are commonly provided through character customization but also branching narratives.
Serious games should have a special focus on ensuring that players autonomously progress into more difficult training activities.

(4) \textbf{Relatedness} occurs when players create social connections with other players or in-game characters.
Multiplayer games inherently satisfy this need using team cooperation or friend networks.
However, singleplayer games are able to rely on virtual characters with whom the player can form relationships.

\subsection{Extrinsic Motivation}
Extrinsic motivation provides separate, desirable outcomes to the player which can be realized using explicit rewards \cite{Birk2016}.
However, psychological principles apply as well.
Four approaches increasing extrinsic motivation are identified: (1) external regulation, (2) introjection, (3) identification, and (4) integration.

(1) \textbf{External regulation} through the means of rewards is a common approach.
Rewards are beneficial for players in a broader scope but they are separate from the objectives players are pursuing.

(2) \textbf{Introjection} is achieved when players receive approval from others or even themselves.
This is implemented using rare status items or hard to earn ranks and achievements.
These introjection elements may not have a direct use in the game but they carry prestigious value creating a feeling of importance.

(3) \textbf{Identification} reinforces self-endorsement of activities with a greater goal.
Seemingly dull and repetitive tasks are accepted when players see the value in completing these tasks.

(4) \textbf{Integration} describes motivation created by aligning the goals of players with their self-view.
Internal beliefs are the driving factor for taking on challenges.

\subsection{Cognitive Bias Modification}
Disrupting addictive behavioral patterns, specifically in regard to substance abuse, is a major challenge in society \cite{Boendermaker2015}.
It has been shown that intervening during the formative period of adolescence is an effective measure to prevent later negative developments.
Interventions fall into one of two categories by being explicit or implicit.
Explicit interventions can utilize clear warning messages or personalized interview which directly touch upon the topic in question.
The effectiveness of explicit techniques has been criticized pushing towards implicit approaches such as cognitive training.
Addictive behavior is often triggered due to strong impulsive reactions overriding relatively weak control processes.
Cognitive Bias Modification (CBM) aims to reduce the imbalance by providing more time for decision making and strengthening the control processes in general.
In practice, CBM is already realized using computer-based reaction time tasks.
As training involves extensive, repetitive practice sessions adding game elements is a viable approach for making the technique more entertaining.
Selective attention is a typical training area for CBM which relies on Visual Probe Tasks (VPT).
Six steps work towards gamification of this implicit, cognitive technique \cite{Boendermaker2015}:
\begin{enumerate}
    \item As a first step, simple game elements are added to the evidence-based task. They are of extrinsic motivational nature and thus usually reward systems. Audible and visible feedback responses are another way of encouraging players to progress.
    \item Intrinsic motivation is the next crucial step in order to make the training task itself more enjoyable. External rewards may be able to keep participants in the loop but transforming the training task into an entertaining game proves to be more effective.
    \item Instead of applying intrinsic motivational features to an existing training task it is possible to start with a game that adheres to the principle of an evidence-based task. However, it might be more difficult to identify which bias modification contributed to the desired effect. While this does not prevent the training game from being effective it complicates research.
    \item Adding a game-shell around the training task is a further, progressive approach. The experience resembles a typical game structure while the original training task remains unmodified. Actual virtual environments are an advantage of this approach as they do not alter the training task but support the gamification goal.
    \item This step combines the intrinsic integration approach with the game-shell. As the game-shell alone can be interpreted as an extrinsic motivator adding intrinsic motivators leads to a better result. Nonetheless, this combination is arguably difficult to implement as it is a threefold integration of CBM ideas with intrinsic motivators as well as the extrinsic reward system of the game-shell.
    \item Finally, off-the-shelf (OTS) games may be used as the starting point for improving selective attention. Games stemming from the entertainment industry naturally fulfill the motivational aspect. However, integrating the CBM specific topics into existing OTS games proves to be difficult or even unfeasible for a range of games.
\end{enumerate}
\section{Enhanced Gaming} \label{sec:enhanced-gaming}
The traditional gaming experience can be enhanced using different techniques focusing on various aspects of the experience.
Adaptive games (section \ref{sec:adaptive-games}) are concerned with generating spaces and missions depending on the player.
Furthermore, the player profile is exploitable for adjusting the difficulty in a dynamic way as well.
Virtual reality (section \ref{sec:virtual-reality}) provides a fully immersive visual experience which incorporates the user's position and orientation.
Finally, haptics provide tactile feedback (section \ref{sec:haptics}) being a crucial building brick for advancing virtual reality from the visual and aural experience.

\subsection{Adaptive Games} \label{sec:adaptive-games}
Procedurally generating content has proven to be a popular method for a range of games \cite{Dormans2011}.
Starting early with the text-based Rogue (1980) dungeon crawler and nowadays commonly known in Minecraft (2009).
While being successfully applied in these cases the technique is not directly suitable for generating action-adventure games.
More specifically, incorporating established level-design principles such as flow, pacing, and learning-curves is challenging.

It is naive to assume that level generation for adventure games is an exclusively spatial problem \cite{Dormans2011}.
Levels are essentially spaces built around structured missions.
Missions, in turn, are non-linear sequences of tasks players need to complete.
This relationship indicates that generation can be split into distinctive processes for missions and spaces.
They are virtually independent apart from their many-to-many relationship: one mission can be mapped to multiple spaces and one space can accommodate multiple missions.

Generative grammars are a concept stemming from linguistics and furthermore usually used in computer science for defining programming languages.
Using an alphabet of symbols and a set of rewrite rules a grammar is able to produce all valid phrases of its language.
In their core grammars work by replacing symbols on the lefthand side of a rule with a single symbol or a group of symbols on the righthand side \cite{Dormans2011}.
This process continues until all rules are replaced by terminal symbols which do not have a matching rewrite rule.
The concept of generative grammars can be translated to game concepts as well.
Once applied to levels symbols in the alphabet become game-specific concepts such as obstacle, enemy, treasure, lock, and key.
If the grammar contains multiple matching rewrite rules for a symbol a random selection can be made.
The advantages and disadvantages of using generative grammars for content generation in games can be found in table \ref{tab:grammar}.

\begin{table}
\begin{tabularx}{\linewidth}{>{\parskip1ex}X@{\kern4\tabcolsep}>{\parskip1ex}X}
\toprule
\hfil\bfseries Pros
&
\hfil\bfseries Cons
\\\cmidrule(r{3\tabcolsep}){1-1}\cmidrule(l{-\tabcolsep}){2-2}

Automation of some design tasks\par
Very controlled generation

&

Result is difficult to foresee\par
Construction of grammar is expensive

\\\bottomrule
\end{tabularx}
\caption{Pros and cons of generative grammar for games}
\label{tab:grammar}
\end{table}

Graph grammars are a specialized extension of generative grammars with the main difference being that they produce graphs consisting of vertices and edges.
The basic principle for the recursive rewrite operation stays the same.
Missions can be represented as graphs with a start, goal, and a varying number of tasks in between.
The linearity of such simple graphs is not applicable for most action-adventure games.
Therefore, graph grammars are able to reorganize linear mission graphs into their non-linear counterpart (see figure \ref{fig:mission-graph}).
Multiple branches in a mission are commonly realized using slightly differing ``lock and key''-tasks \cite{Dormans2011}.

\begin{figure}[H]
    \centering
    \includegraphics[width=\linewidth]{assets/mission-graph.png}
    \caption{Non-linear mission graph with tasks (T), locks (L), and keys (K) \protect\cite{Dormans2011}}
    \label{fig:mission-graph}
\end{figure}

In a straightforward way, locks represent obstacles requiring a specific key which is only available in a different part of the mission.
This also highlights the controllable strength of grammars. 
Given that the graph grammar is defined correctly it is automatically ensured that missions are solvable by organizing its elements in a valid order.
Furthermore, it is relatively simple to specify the number of times a certain task occurs and how often rules are applied.

With a structured mission in place, multiple approaches can be used to generate spaces.
Similarly to generative and graph grammars, special shape grammars are able to create spaces by following rewrite rules for shapes.
The inability of shape grammars to let multiple paths converge on the same target is remedied using an organic spring-based layout \cite{Dormans2011}.
Before the shape grammar is applied the mission graph runs through a simulation where the edges between vertices act as springs (see figure \ref{fig:organic-mission-layout}).
Overlapping edges are removed from the resulting organic layout.

\begin{figure}[H]
    \centering
    \includegraphics[width=5cm]{assets/organic-mission-layout.png}
    \caption{Organic mission graph layout \protect\cite{Dormans2011}}
    \label{fig:organic-mission-layout}
\end{figure}

Finally, player models are the crucial required component for creating adaptive games.
In their essence, they represent a player's behavior in a game.
Three distinctive approaches are identified for player modeling \cite{Dormans2011}:
\begin{itemize}
    \item \textbf{Action modeling} is used to predict actions players might take in specific situations.
    \item \textbf{Preference modeling} aims at identifying the preferences of a player leading to certain actions.
    \item \textbf{Player profiling} goes beyond modeling in the sense that it automatically creates psychologically verified player profiles. It works towards extracting traits of a player's personality.
\end{itemize}
Subsequently, player models can be exploited for adapting spaces, missions, and difficulty in games.
Features from the player model provide information for transforming the surrounding environment or changing gameplay through mission growth.
Naturally, having knowledge about the player model allows for dynamic adaption of a game's difficulty.
Usually, it is desirable to achieve a balance between the player's skills and challenge in the game.
Nevertheless, it could be used to create intentional frustratingly difficult games as well.

Being able to dynamically adapt various areas of a game is not only useful for entertainment purposes \cite{Lopes2011}.
Serious training games benefit from detailed player profiles as well.
It enables more unique and personalized experiences.
Classifying the affective state of players is not necessarily restricted to in-game actions alone.
Progression in the field of affective computing and recognition of facial, motion, and physiological features provides valuable input for adaptivity.
This allows identifying more affective states such as fun, frustration, predictability, anxiety, and boredom.
Concerning different areas of focus in research and industry efforts adapting missions and spaces is still an issue to be progressed in the future.

Detecting and recognizing facial expressions is a rather advanced field in computer vision \cite{Blom2014}. 
While this is the case actual usage for user experience analysis in non-game applications is being done in a limited number of applications.
As mentioned before relying on facial expressions for the adaption of game content is a viable solution.
It can be implemented in an unobtrusive way and in an online scenario, meaning that adaption happens while the user is playing the game.
The objective of the following scenario is to match the game's challenge with the current affective state of the player.

Facial expressions are continuously identified outputting probability distributions for seven distinct emotions: (1) neutrality, (2) happiness, (3) disgust, (4) anger, (5) fear, (6) sadness, and (7) surprise \cite{Blom2014}.
When recognition of facial expression is applied to online level generation there are two events when the affective state is used for adaption: when a player nears the end of a segment or in the case of death resetting the player to the beginning.

The Gradient Ascent Optimisation (GAO) technique is used to select appropriate level challenges.
Its objective is to minimize negative emotional responses while maximizing positive reactions.

Only neutrality, happiness, and anger are taken into account by the proposed system.
Furthermore, the authors point out that other non-verbal and verbal cues have to be recognized as well \cite{Blom2014}.
In their experimental tests, participants expressed anger using gestures and verbal actions.
\subsection{Virtual Reality} \label{sec:virtual-reality}
Virtual reality (VR) has garnered a lot of attention and hype in recent years.
It is regarded as a novelty which naturally spikes interest in people.
However, technical limitations still prevent mainstream adoption.
This leaves consumers disappointed and somewhat disillusioned about the hype in VR.
Interestingly enough the concept of VR has been established for quite some time ago with Ivan Sutherland's vision in 1965 \cite{Brooks1999}.
Working implementations have been published in 1994 with the first production systems in 1999 already.

Four technologies are crucial components for VR systems and essentially encompass the definition \cite{Brooks1999}:
\begin{itemize}
    \item Visual, aural, and haptics displays immersing the user in a virtual world. Contradicting sensory signals from the real world are blocked out.
    \item The graphics rendering systems constantly streaming images with at least thirty frames per second to simulate a fluid reality.
    \item The tracking system providing information about the position and orientation of the user's head and extremities.
    \item The system responsible for building and maintaining high fidelity of the virtual world to create a realistic experience.
\end{itemize}
While these components provide the basics for VR additional technologies are able to improve the immersive feeling including synthetic directional sound and sound fields, tactile and kinesthetic feedback (see also section \ref{sec:haptics}), specialized tracking gloves enabling better interactions with the virtual world, and finding adequate replacement interactions for their counterparts in the real world.

Multiple areas proved to be challenging for the usage of VR in production environments: rendering engines, tracking, ergonomics, and latency \cite{Brooks1999}.
The most problematic issues are solved to an acceptable degree which allowed the first consumers to get in touch with VR systems.
However, specific issues in rendering engines continue to exist and especially the ergonomics of equipment requires further improvement.
Exponentially growing power of graphics processing units (GPU) in recent years solved the most pressing matter of rendering engines.
Head-strapped displays lacking retina resolution carry the inherent problem that users are able to see the underlying grid due to the eyes being in close proximity to the display surface.
This issue will certainly be rendered obsolete with the fast-paced progression of display technology.
Ergonomics are still problematic since powerful VR systems require a wire from the worn equipment to the computational unit.
Wireless systems certainly exist but lack far behind the capability of dedicated desktop stations.
\subsection{Haptics} \label{sec:haptics}
While the world continues to being transformed by digitalization a lacking area is haptics of virtual objects.
Tactile technology has progressed with large companies such as Apple embedding the \emph{taptic engine} in all of its recent devices in order to provide tactile feedback under the marketing name \emph{force touch}.
However, the majority of feedback devices require an interactive surface or wearable haptic equipment.

AIREAL is a novelty tactile feedback system utilizing precise air vortices to create expressive sensations \cite{Sodhi2013}.
In combination with interactive displays or virtual reality (see also section x.x) it bears the potential of allowing users to feel 3D virtual objects without touching or wearing additional haptic equipment.
This fosters blurring the line between the virtual and real world.

Air vortices are created through pressure differences in a relatively small cube with a nozzle that can be freely actuated.
Their advantages are a relatively long travel distance (up to one meter), efficiency, low costs, and scalability.

Five principles steered the design of applications for the AERIAL haptics system \cite{Sodhi2013}:
\begin{itemize}
    \item \textbf{Collocation:} Visual images and tactile sensations should be collocated in space and time in the sense of an overlap between projected images and haptic feedback on the user's body.
    \item \textbf{Persistence:} Static areas and objects are able to emit fixed tactile simulations representing real physical objects.
    \item \textbf{Variance:} The haptic sensations can represent varying, rich textures in 3D space.
    \item \textbf{Continuity:} The tactile feedback can be moved continuously around the user.
    \item \textbf{Transience:} Air haptics are able to actuate real physical objects in the environment near the user.
\end{itemize}

The authors identify two main limitations of the system \cite{Sodhi2013}.
Firstly, an audible knock is produced due to the use of a high amplitude and low-frequency signal.
Secondly, while it achieves its goal of avoiding active instrumentation for the user it still requires passive instrumentation for operations in the user's environment.
\section{Ontology} \label{sec:ontology}
\citeauthor{Elverdam2007} (2007) argue that the two common ways of describing games, namely by direct comparison or genre categorization, are not sufficient for a precise classification.
Direct comparison usually relies on one specific, similar connection but does not account for all the other differences.
Assigning games into genres works for a broader categorization but it also falls short of identifying inter-genre differences.
The authors iterate on top of the open-ended typology by \citeauthor{Aarseth2003} (2003) which contains sixteen dimensions grouped into six meta-categories.
They propose a modification and extend the meta-categories to eight in total: (1) virtual space, (2) physical space, (3) internal time, (4) external time, (5) player composition, (6) player relation, (7) struggle, and (8) game state.
Each meta-category features deeper dimensions for further classification.

\textbf{Virtual Space:}
\begin{itemize}
    \item \textbf{Perspective:} In an omnipresent perspective the player is able to see a total overview while a vagrant perspective requires strategic movement.
    \item \textbf{Positioning:} Absolute positioning allows to precisely discern the player's position. Alternatively, relative positioning creates a dependency between the player's position and other objects.
    \item \textbf{Environment:} Three different ways for the player to make modifications to the game space: (1) free, (2) fixed, or (3) none.
\end{itemize}

\textbf{Physical Space:}
\begin{itemize}
    \item \textbf{Positioning:} Omnipresent allows viewing the entire physical game space while vagrant requires movement.
    \item \textbf{Positioning:} The player 's position can be relatively determined in relation to other game agents (proximity based) or both factors combined.
\end{itemize}

\textbf{External Time:}
\begin{itemize}
    \item \textbf{Teleology:} Games can end after a finite amount of time or carry on infinitely.
    \item \textbf{Representation:} Whether time mimics reality or behaves arbitrarily.
\end{itemize}

\textbf{Internal Time:}
\begin{itemize}
    \item \textbf{Haste:} Present haste creates a connection between the game state and real-time. Absent haste indicates no such connection.
    \item \textbf{Synchronicity:} If it is present game agents are able to act at the same time.
    \item \textbf{Interval Control:} With present control, players can trigger the next game cycle on their own.
\end{itemize}

\textbf{Player Composition:}
\begin{itemize}
    \item Different types of player organization: single player, single team, two players, two teams, multiplayer, or multiteam.
\end{itemize}

\textbf{Player Relation:}
\begin{itemize}
    \item \textbf{Bond:} Player relations can change dynamically or are static.
    \item \textbf{Evaluation:} Different forms of outcome quantification: individual players, as teams, or both.
\end{itemize}

\textbf{Struggle:}
\begin{itemize}
    \item \textbf{Challenge:} Three ways of opposition in games: (1) identical are predefined challenges which are always the same, (2) instance uses a predefined framework which varies through randomness, or (3) agents which act autonomously.
    \item \textbf{Goals:} Games can have absolute goals which remain fixed or relative towards the unique events in a game session.
\end{itemize}

\textbf{Game State:}
\begin{itemize}
    \item \textbf{Mutability:} Changes in the game state can be temporal, finite for a game session, or infinite by spanning multiple game sessions.
    \item \textbf{Savability:} The conditions for the player to save and restore the game state being either unlimited, conditional or no savability.
\end{itemize}

The Game Ontology Project (GOP) as presented by \citeauthor{Zagal2007} (2007) is a fully fledged framework for describing, analyzing, and reasoning about games.
It is a more detailed approach in comparison to the classification model and aims to identify structural game elements as well as their relationships in a hierarchical organization.
Game design spaces are characterized by focusing on components that cause, effect, and relate to actual gameplay.
The GOP uses concepts from prototype theory in combination with a methodological, inductive approach from grounded theory to generate the ontology.
In order to characterize game design spaces the representational specifics of games are represented in an abstract way in this framework.
Five top-level elements constitute the ontology: (1) interface, (2) rules, (3) goals, (4) entities, and (5) entity manipulation.

\begin{itemize}
    \item \textbf{Interface:} This is the bi-directional communication area between the player and the game. It allows players to take in-game actions by actuating input devices of various forms. Physical signals are mapped to digital actions in the game. In the reverse direction, the game provides sensory feedback to the player usually in the form of visual graphics and aural sensation.
    \item \textbf{Rules:} Define the constraints and level of freedom in a game. They describe how a game unfolds and the possible interactions within. Rules can be further differentiated into gameplay and game world rules. The latter affects the whole game world and are commonly used in simulator type games. Gameplay rules lay down the specifics of game flow such as the abilities and meta attributes of players.
    \item \textbf{Goals:} Goals encompass in-game objectives that must be accomplished in order to win. While they are clearly defined they are not necessarily communicated to the player. Goals can be analyzed at different levels as there are overarching objectives for games which contain subordinate goals which are required for their completion.
    \item \textbf{Entities:} The objects in the reality of the game world. It is emphasized by the authors that this section requires future development. This is justified by the fact that entities are implicitly defined by entity manipulations.
    \item \textbf{Entity Manipulation:} Entities have attributes and abilities which are altered by entity manipulations. In this sense, abilities are the verbs of entities which are the nouns of the game world. Subsequently, attributes are the adjectives being changed by abilities. Entities lacking abilities are categorized as being static as they do not manipulate their environment. In contrast, dynamic entities use their abilities to change attributes of other entities.
\end{itemize}

\subsection{Ethics}
\citeauthor{Sicart2005} (2005) discusses the importance and often overlooked aspect of games as ethical systems.
Existing ontologies describe games using all their structural properties including visuals, narratives, and interactions which constitute the experience.
However, the fictional level of games should be analyzed as well in order to work towards better understanding.
Digital games feature the special property of enforcing a set of rules
These rules are not subject to discussion and are encapsulated in a kind of black box that usually refrains from revealing its inner processes.
Naturally, virtual worlds are compared to the real world as a point of reference.
Rules constraining possibilities within games constitute an ethical system which is evaluated by players.
Reflecting on games as ethical objects allows to pose interesting questions in the range of ``To what extent does it matter what happens in fictional, virtual worlds?'', ``Are game designers responsible for making ethical restrictions?'', ``Is it beneficial if non-ethical actions are expressed in virtual worlds?''.
\section{Conclusion}
It becomes apparent when reviewing the literature selected in this paper that games do not exclusively serve their initial entertainment purpose.
Business, training scenarios, and behavioral modification practice all benefit from gamification and serious game.
Entertaining aspects of games, namely intrinsic motivation, is a powerful tool which can be applied to serious scenarios.
Featuring realistic virtual humans is a crucial component, especially for training purposes.
The boundaries of classic gaming are extended through adaptive gaming techniques dynamically generating content relative to the player profile. 
Virtual reality is arguably a field bearing a lot of potential for the mainstream entertainment industry.
Analyzed in the Gartner hype cycle\footnote{\url{https://www.gartner.com/it-glossary/hype-cycle}, accessed 06-04-2018} it currently travels the trough of disillusionment.
Technological advancements, specifically in GPUs, solved the most technical limitations.
However, ergonomics continue to be problematic leaving consumers disappointed after the excitement.

All in all, games continue to grow beyond being media for solely entertainment purposes.
The computational capabilities of modern computer systems allow game designers and researchers to focus on areas distinct from the gameplay itself.

\bibliographystyle{ACM-Reference-Format}
\bibliography{Mendeley}

\end{document}
