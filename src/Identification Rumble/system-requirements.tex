% !TEX root = ./setup.tex

\subsection{System Requirements}
This project design poses a set of requirements for both the museum and its visitors.
For visitors, these requirements are mainly focused on the interactive procedure of going through the exhibition from beginning to end.
In order to realize the project, the museum has to meet administrative and technical prerequisites.
These requirements will be discussed from separate perspectives in the following sections.


\subsubsection{Visitor Requirements}
Visitors are able to pick up an identity card replica, containing an RFID tag, at the entry of the museum.
This is precisely where the museum already hands out audio tour devices.
It should be made clear that the identity card is the only item they need to interact with the exhibition during their visit.
Additional explanations are not necessary which is in line with the initial stakeholder's guidelines.
Naturally, visitors are expected to return their identity card at a designated collection point before leaving the museum.

When interacting with different stations of the exhibition visitors should understand the main interaction of holding their identity card to scanner points.
Holding an identity card within the proximity of an RFID reader is the central action carried through the exhibition.
It either triggers the start of a station or signifies confirmation of a choice made by the visitor.
In addition to that, visitors need to recognize interaction points or namely where identity cards will be recognized.
Understanding this interaction will be supported by visual aids near interaction points and audiovisual feedback from stations.


\subsubsection{Museum Requirements}
The museum has to build props (identity cards), acquire necessary hardware (stations), setup software system and produce digital material (dilemma animations) to realize the project.

Identity cards are set together by equipping a physical replica with an RFID tag.
Therefore, custom replicas have to be assembled mimicking the look and haptic of identity cards used during WWII.
Quality of these props can range from simple paper printouts to elaborate duplicates using leather covers.
This can be made dependant on the available budget.
On the technical side, RFID tags are easily acquired through local or online retail businesses.

Each station requires a specific hardware setup including RFID readers, screens and computers.
Language choice is backed by a single computer connected to an RFID reader per supported language of the museum.
Dilemma stations rely on a screen for playing their respective animation, a computer and an RFID reader per answer of the dilemma.
The evaluation station consists of a screen, computer and single RFID reader recognizing visitors with their identity card.

As the stations are at different locations within the museum one backend server acts as the communication hub.
This server needs to be setup one time by pulling the project's repository\footnote{\url{https://github.com/marc1404/identification-rumble}, Accessed: 09-02-2018} and installing Node.js\footnote{\url{https://nodejs.org/en/}, Accessed: 09-02-2018} as the runtime framework.
Needless to say, it should be kept running to render the exhibition stations usable and administrative interfaces available.
In order to connect the stations to the server, an internal network has to be provided.
It is not advisable to open the server to the public internet as this would open it up to malicious attacks.

Finally, each dilemma features an animated video with audio which guides visitors through the scenario.
These clips have to be produced from a technical side but also require voice actors to record narration and dialogue.
Furthermore, they should be available in all languages which the museum seeks to offer the exhibition in.
