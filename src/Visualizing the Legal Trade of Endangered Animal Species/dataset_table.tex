\begin{table}[!ht]
\begin{tabular}{ l l p{8cm} }
\hline
\emph{Name} & \emph{Type} & \emph{Description} \\
\hline
Year & Integer & Year the export or import occurred \\ \hline
Appendix & String & Varying from I to III. I is most protected, III the least. \\ \hline 
Taxon & String & The species' taxonomic taxon \\ \hline 
Class & String & The species' taxonomic class \\ \hline 
Order & String & The species' taxonomic order \\ \hline 
Family & String & The species' taxonomic family \\ \hline 
Genus & String & The species' taxonomic genus or scientific name \\ \hline 
Importer & String & Abbrevation of the country in ISO-3166-1 importing the species \\ \hline 
Exporter & String & Abbrevation of the country in ISO-3166-1 exporting the species \\ \hline
Origin & String & Abbrevation of the country in ISO-3166-1 which the species originated \\ \hline
Import reported quantity & Integer & Quantity of the imported species' as reported to CITES. Blank if an export record. \\ \hline
Export reported quantity & Integer & Quantity of exported species' as reported to CITES. Blank is an import record. \\ \hline
Unit & String & Quantity type varying from 'g' for grams to 'lbs' for pounds. Default is unit quantity. \\ \hline
Purpose & String & The purpose with which the species was exported or imported. \\ \hline
Source & String & Describes how species was brought onto market. \\ \hline
\end{tabular}
\caption{Fields in the Wildlife dataset as described by CITES} \label{tbl:dataset-fields}
\end{table}