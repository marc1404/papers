\section{Story} \label{sec:story}

Altered Carbon is a dystopian series produced by Netflix and based on the original novels by Richard K. Morgan.
In the fictional future of 2384 humanity wields the power to transfer a person's memory into other bodies at any time.
Additionally, the process is not restricted to organic bodies alone since they can be synthetically fabricated as well.
Due to this leap in technological progress bodies merely serve the purpose of carrying consciousness.
The society in Altered Carbon coined the term ``sleeves'' as a physical body can virtually be changed in a similar fashion to regular clothes.
Disposable Sleeves derives its title and background story from the reality depicted in the series.
This allows the creation of a bounding framework for the gameplay which provides an explanation to the question: "Why do players find themselves in a trap-race scenario with one person in full control of trap activation?".

The process of re-sleeving a person's consciousness into a different sleeve incurs a high cost measured in United Nations credits.
Depending on the configuration of the sleeve (i.e. physical characteristics) the costs can be even higher.
In a true capitalist fashion, the richest 1\% of society, called \emph{meths}, is able to afford any sleeve and is virtually immortal.
Furthermore, the concept is taken to a dark extreme as \emph{meths} use other people for their entertainment in e.g. duels.
Since physical death has no meaningful consequence this is considered legal, provided that \emph{meths} cover the re-sleeving costs.

Therefore, the trap master in Disposable Sleeve represents a wealthy \emph{meth} whereas other players are essentially lab rats who can be re-sleeved in case of their death..
Running players have a common objective of reaching the end but they do not necessarily cooperate.
The first player to reach the end usurps power from the current \emph{meth} and repeats the cycle.