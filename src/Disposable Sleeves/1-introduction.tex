\section{Introduction} \label{sec:introduction}

\emph{Deathrun}\footnote{\url{https://www.urbandictionary.com/define.php?term=Deathrun}, accessed 29-03-2018} is multiplayer game mode existing within the realm of existing games such as Roblox (2006)\footnote{\url{https://www.roblox.com/}, accessed 29-03-2018}, Valve's Counter-Strike series (2000, 2004 and 2012)\footnote{\url{http://store.steampowered.com/search/?term=counter-strike}, accessed 29-03-2018} and Facepunch Studios' Garry's Mod (2004)\footnote{\url{https://gmod.facepunch.com/}, accessed 29-03-2018}.
In general two teams, deaths and runners, compete against each other in courses filled with traps.
Runners have to reach the end of a level in order to defeat deaths.
Deaths try to prevent the runners from achieving their goal by activating traps along their path.
After more than 10 years since the majority of these games have been published deathrun still proves to be popular amongst players all over the world.\footnote{\url{https://www.gametracker.com/search/?search_by=map&query=deathrun}, accessed 29-03-2018}
However, it has not been published as a standalone game yet.
This fact motivated the conception of \emph{Disposable Sleeves} a first-person multiplayer game built on the idea of original deathrun maps.
Furthermore, the gameplay is wrapped into the dystopian future of \emph{Altered Carbon} as envisioned in Richard K. Morgan's novels (2002)\footnote{\url{https://www.richardkmorgan.com/books/altered-carbon/}, accessed 29-03-2018} and the television adaption of Netflix (2018)\footnote{\url{https://www.netflix.com/title/80097140}, accessed 29-03-2018}.
This paper aims to provide a factual description of the game project, at the time of writing, through the sections Gameplay (section \ref{sec:gameplay}), Story (section \ref{sec:story}) and Technical Implementation (section \ref{sec:implementation}).
Finally, section \ref{sec:conclusion} recaps the motivation behind and value of the game.
It also describes the differences between initial goals and current realisation thereby identifying future work items.