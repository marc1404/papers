\section{Conclusion \& Future Work} \label{sec:conclusion-future-work}
The proposed \textit{Beerlytics} system succeeds in generating visual stories of beers while exploiting vector space models of their reviews trained by different word embedding algorithms.

In order to build its dataset, the system scrapes relevant links from \textit{RateBeer} and scrapes their corresponding pages thereafter.
The data collection process persists the data to a MySQL database that is shared by all researchers working on the larger \textit{Beerlytics} platform.

Subsequently, data analysis transforms raw data into valuable information through multiple steps.
Raw data is cleaned and preprocessed before being fed into the different word embedding algorithms.
This process trains vector space models for each beer and algorithm in use.
Clustering techniques are applied the resulting models which identify representative and diverse reviews.
Additionally, keyword extraction provides insights into a beer product from the angle of different aspects.

Finally, the frontend prototype turns the output from the computation process into visual product stories.
It communicates with the API to satisfy its data requirements.
Evaluation of the frontend prototype in 2 iterations with 6 participants in each round tests the effectiveness of the systems and gathers useful feedback.

Considering all findings, participants prefer \textit{Beerlytics} over the original \textit{RateBeer} platform.
However, the system leaves room for improvement in regards to the summarization aspect of the page and ensuring a high standard for the quality of extracted keywords.

\hfill

\noindent
Future work includes the implementation of suggested improvements, exploring the problem using quantitative methods, and potential benefits for interactive multimodal learning scenarios.

Firstly, multiple participants in both user testing rounds recommend a sentiment scoring for extracted keywords.
As they currently stand, it is challenging to judge whether a positive or negative context surrounds a keyword.
Realization requires localizing the keywords in their original review and performing sentiment analysis on the context.
Given the localization of keywords, they could also be linked to the review component and highlighted in the text.

Secondly, applying the same approach for training models from beer reviews to breweries and also users is a future possibility too.
The models for breweries and users would allow their exploration and potentially comparisons between them.

Thirdly, a highly requested functionality during user testing is regional filtering of the dataset (e.g., restricting the data to the Netherlands only).
This regional filtering would require computation on varying regional levels.

Fourthly, quantitative and automated methods could potentially be used for the evaluation.
Currently, no ground truth data exists about the quality of reviews in the dataset.
Crowdsourcing platforms like \textit{Amazon Mechanical Turk (MTurk)}\footnote{\url{https://www.mturk.com}, accessed: 17-07-2018} offer human intelligence through programmable interfaces.
Utilization of human intelligence could establish the necessary ground truth data by creating human intelligence tasks (HITs) judging the quality of reviews.

Finally, the compact representations generated by the system could benefit interactive learning scenarios where quick relevance judgment of an item is necessary \cite{Zahalka2015, Zahalka2018}.
Representation of an item should be as condensed as possible while most of the original information content is preserved.