\section{Introduction}
The widespread adoption of e-commerce has established consumer-generated reviews and ratings of products as a standard feature on merchant websites.
These reviews provide guidance for potential new customers and help to meet their expectations after receiving an ordered product.
Besides creating a knowledge database for consumers reviews contain valuable insights and measurements for both merchants and producers.
While not every single customer authors a review especially popular products amass numerous reviews on large marketplaces during their life cycle.
Reviews themselves differ in length, level of detail, quality of writing, structure, language, and opinion features \cite{Hu2004a}.

However, e-commerce platforms of producers, merchants and large-scale retailers are by no means the only place for users to express their opinion about products.
Review aggregation sites collect user reviews about products and services as a third party which does not offer the reviewed subjects themselves.
Well established household names include: \textit{Metacritic}\footnote{\url{http://www.metacritic.com}} (entertainment), \textit{Rotten Tomatoes}\footnote{\url{https://www.rottentomatoes.com}} (movies, tv), \textit{IMDb}\footnote{\url{http://www.imdb.com}} (movies, tv, actors) and \textit{TripAdvisor}\footnote{\url{https://www.tripadvisor.com}} (vacation, flights, restaurants).
Aggregators act as independent platforms where users can find averaged reviews in a uniform fashion.

As a more specific instance with a much more narrow focus \textit{RateBeer}\footnote{\url{https://www.ratebeer.com}} collects information about (craft) beer and primarily consumer-generated reviews of beers and breweries.
Established in 2000 it has since remained a popular exchange platform garnering more than five million reviews in total\footnote{\url{https://www.ratebeer.com/RateBeerBest/default_2013.asp}, accessed: 16-03-2018} with a rate of roughly 1 million reviews per year since 2016\footnote{\url{https://www.ratebeer.com/ratebeerbest/default_2016.asp}, accessed: 16-03-2018}.
While more recent and exact figures remain hidden it is estimated that nearly 500,000 visitors view the site within a single month\footnote{\url{https://www.similarweb.com/website/ratebeer.com}, accessed: 16-03-2018}.

Due to the following factors, \textit{RateBeer} acts as a testbed for this thesis project.
Firstly, the large quantity of historical reviews and the constant stream of new content provides an abundance of data for processing and experimental testing.
Secondly, according to \textit{RateBeer's} quality assurance principles, low quality and nonsensical entries are swiftly removed from the platform through an extensive administration system.
Furthermore, although \textit{Anheuser-Busch InBev} acquired a minority stake of \textit{RateBeer}\footnote{\url{https://www.nytimes.com/2017/06/18/business/media/anheuser-busch-inbev-ratebeer.html}, accessed: 16-03-2018} the site claims and emphasizes its mission to remain an independent platform which prohibits ratings by breweries and their affiliates.
Finally, reviews are of a semi-structured nature containing free-form text but also scaled ratings in 5 distinct categories: (1) aroma, (2) appearance, (3) taste, (4) palate, and (5) overall.

The following parts of this paper are structured as follows.
Firstly, section \ref{sec:related-work} examines related work in the fields of word embeddings and (visual) summarization.
Secondly, section \ref{sec:problem-statement} contains the problem statement alongside the pursued research questions.
Thirdly, section \ref{sec:method} details the approach taken for the realization of the \textit{Beerlytics} system going from data collection and analysis, over the system design of the API and frontend prototype, to the final evaluation.
Fourthly, section \ref{sec:discussion} discusses actual and potential shortcomings of the research project.
Finally, section \ref{sec:conclusion-future-work} concludes this research by recapping the proposed system and laying out recommendations for future work.