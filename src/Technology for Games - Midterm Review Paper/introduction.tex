\section{Introduction}
The goal of this review paper is to establish a broad birds-eye view on the following areas within games: Design frameworks, mechanics, story, and artificial intelligence.
In the beginning the three main design frameworks are covered chronologically.
Published in 2004 the Mechanics-Dynamics-Aesthetics framework (MDA) by Hunicke et al. \cite{Hunicke2004} set out to establish a common language to reason about games and analyze them from different perspectives (Section \ref{sec:mda}.
The elemental tetrad by Schell in 2008 \cite{Schell2014} picks up the concepts of mechanics and aesthetics (Section \ref{sec:elemental-tetrad}).
It introduces story and technology and puts the components into relation.
This paper continues by covering mechanics (Section \ref{sec:mechanics} and story (Section \ref{sec:story}) as identified by Schell in greater detail.
He argues that while it is difficult to create a unified taxonomy due to the hidden complexity of mechanics it is still worthwile to work towards a common understanding.
Mechanics are the core of every game and need to be understood to reason about it.
Interactive storytelling is a new concept introduced by videogames which proves to be inherently difficult to implement.
Paying attention to traditional storytelling and following guidelines for the creation of the feeling of freedom helps to produce interesting experiences using interactive storytelling.
Finally, the last section of the paper provides insights into the ten research areas of AI in games as Yannakakis and Togelius identified them in 2015 (Section \ref{sec:ai}).
It also explores the application of AI planning to computational narratives in the article by Porteous et al.\cite{Porteous2010} (Section \ref{sec:search-and-planning}).