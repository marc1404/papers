\subsection{Story} \label{sec:story}
Historically speaking, story and gameplay have been two separate things.
Stories follow a single narrative whereas games can host groups of players and lead to different outcomes.
With the introduction of computer games to mainstream entertainment the lines between story and interactive gameplay started to blur.
Effectively challenging the assumption that they cannot be combined in its grounds.
Naturally, there are still ongoing debates about the interplay of story and gameplay.
Shifting the focus to published games shows that they consistently feature story elements of various kinds.
There are games which are build around a strong story line and in contrast less obvious stories intended for discovery and self-exploration.
\citeauthor{Schell2014} boils this combination down to a tool which helps to create and shape experiences.
The author explicitly stresses the similarity between traditional storytelling and its interactive variant.
They are not inherently different as both induce a desire to take action in listeners.
However, the latter actually enables listeners to participate and storytellers have to account for different narrative branches.
The following section discuss methods (\ref{sec:story-methods}) for story creation, common problems  (\ref{sec:story-problems}), guidelines for better stories  (\ref{sec:story-design}), and finally indirect control methods to create the feeling of freedom  (\ref{sec:story-freedom}).

\subsubsection{Methods} \label{sec:story-methods}
Interactive storytelling in games is mainly governed by two methods: (1) String of pearls and (2) story machine.

\textit{The string of pearls} (also known as \textit{rivers and lakes} \cite{Schell2014}) method presents a rather non-interactive story along its string.
Choosing techniques to present the story is hereby up to the game designer and can be realized in text, images or animated scenes.
Pearls along the string of story are sequences where players take full control.
They feature one or more goals with one final objective which upon completion progresses the story.
It is commonly criticized for seeming pseudo-interactive but in reality it provides players with a carefully crafted story with periods of interaction and freedom in between.
According to characteristics of interesting goals (Section \ref{sec:mechanics-rules}) the reward for finishing an interactive sequence is clearly the continuation of the story followed by more challenges.

In contrast to a pre-defined story the second method \textit{the story machine} postpones story generation until a game is played.
Good stories can be broken down into a sequence of interesting events \cite{Schell2014}.
Therefore, good games are essentially story machines producing series of events forming an interesting narrative.
This implies that designers are not necessarily defining the heart of these stories ahead of time nor are they in control of what players will actually experience.


\subsubsection{Problems} \label{sec:story-problems}
Creating interactive story trees is rather simple if qualities such as suspense, coherence, and enjoyment are ignored.
The special properties of branched narratives and the very nature of videogames in regards to reality introduce a range of obstacles.
\citeauthor{Schell2014} explores five of these problems in detail:

\begin{enumerate}
    \item \textbf{Unity:} Stories start out with a main problem in the beginning and finish with touching upon this again in the end (most of the time solving the problem). This introductory problem is the force pushing along the story. If players are able to deviate from this or avoid it entirely writing multiple endings with unity in respect to the beginning becomes an immense challenge. Consequently, interactive stories may loose their strength and feel disconnected to the beginning.
    \item \textbf{Combinatorial explosion:} Each choice leading to unique outcomes contributes to the exponential growth of branches within a story. Increasing the depth of branching leads to an excessive amount of story lines. Ensuring quality for every different outcome is not feasible anymore. As a remedy to this problem branches can be merged together again. However, this creates a chain of compromises ultimately rendering the choices meaningless as they lead to a common ending.
    \item \textbf{Multiple endings disappoint:} At first it may seem like providing players with multiple endings creates a rich game. Playing it again will yield a different experience. However, players commonly find themselves questioning whether they took the right path. This is due to alternative endings not having the same level of unity with their beginning. Furthermore, players will dread missing out on other endings and being required to complete the story once again.
    \item \textbf{Not enough verbs:} Activities in videogames vastly differ what characters in book or movie stories are doing. While game technology has seen immense improvements in terms of fidelity recreating natural conversations still feels rather restricted. Stories heavily rely on communication between characters. Progress in the field of artificial intelligence can potentially counteract this problem.
    \item \textbf{Time travel:} Compelling stories usually feature tragedy and inevitability. It is challenging for games to support freedom and control while also supporting inevitability. If the player controls a story's main character, for example, any death will will revert time to the last checkpoint. Designers may choose to increase the punishment by resetting the player further or even to the beginning but this mechanic does not recreate tragic. Nevertheless, sophisticated design is able to circumvent this problem. Examples can be found in the story-driven ``The Walking Dead'' or ``The Wolf Among Us'' games by Telltale Games. Choices of players influence the story in drastic and final ways; going as far as progressing without dead characters.
\end{enumerate}

\subsubsection{Design Guidelines} \label{sec:story-design}
The previously enumerated problems create the impression that interactive storytelling can hardly produce compelling experiences.
\citeauthor{Schell2014} argues that moving the focus from stories back to experiences thereby adjusting expectations should be the goal.
Using traditional storytelling with game mechanics can create unexpected and innovative experiences.
The author suggests nine different guidelines which help to streamline the creation of better game experiences:

\begin{enumerate}
    \item \textbf{Goals, obstacles, and conflicts:} Following the main paradigm of movie storytelling two components play key roles: Characters with goals and obstacles interfering in reaching these goals. Conflicts emerge when other characters aim to achieve contrasting goals. The same principle applied to videogame stories leads to: Players engaging in problem solving and conflicts creating surprising outcomes. Nonetheless, games need to align goals and obstacles with their story. Otherwise they may feel superfluous and have a detrimental effect on the overall experience.
    \item \textbf{Backstory:} While elaborate backstories found in books and movies might not be necessary virtual worlds should have enough support to feel real. They certainly need to feel real for game designers in order to convey this idea to players. Motivation of characters should be grounded in the history of the world.
    \item \textbf{Simplicity and transcendence:} Virtual worlds are simpler than reality while also making the player more powerful than in reality (transcendence). This combination proves to be successful as it allows players to sufficiently understand the virtual world and equips them with the ability to achieve their goals.
    \item \textbf{The hero's journey:} Joseph Campbell describes a common structure of heroic stories in his book. Christopher Vogler turns these fundamentals into a practical guide for creating heroes according to Campbell's identified pattern. Due to the heroic nature of many videogame stories these works are a natural fit for structuring compelling stories.
    \item \textbf{Adapt the story:} It is common to venture out on game design by solely focusing on the story first. This can neglect other essential parts which are often more expensive to change. Balancing gameplay requires fine tuning of mechanics and adaption of technology results in complex reprogramming. Instead of changing rigid game components it might be possible that rewriting specific parts of the story is an easier solution.
    \item \textbf{Consistency:} Game worlds are intricate, fragile constructs which rely on consistency. Only if a world is consistent to itself, meaning that universal rules apply to everything in the same way, players are able to immerse themselves. Even one logical error can initiate the disintegration of the game world as a construct.
    \item \textbf{Accessibility:} Games feature elements that are not necessarily accurate in respect to reality for the benefit of what players will believe and enjoy. It is important to keep surreal and inaccurate elements in mind and finding a way to weave them into the world. Players will accept an unusual world as long as the elements inside are natural to this scenario.
    \item \textbf{Balancing the use of clich\'es:} Designers need to find the right balance between avoiding clich\'es at all costs and overusing them. As a middle ground combining familiar elements with novel additions is a sensible approach. It might provide players with a known clich\'e while introducing something new to build upon.
    \item \textbf{Maps as story elements:} Whether a game occupies a physical space or is described in words, creating maps helps both designers and players to visualize the world. Stories can be draped around the shape of these maps. This approach often leads to natural stories as designers have to think about the actual layout of their worlds.
\end{enumerate}

\subsubsection{Feeling of Freedom} \label{sec:story-freedom}
At its core the clash between story and gameplay is about feeling or more specifically the sense of freedom \cite{Schell2014}.
The power of control is crucial for interactive experiences it allows players to immerse themselves in a world where they are actually able to act in steered by their imagination.
This indicates handing over full control to the player which would create problems discussed in Section \ref{sec:story-problems}.
Meanwhile, giving the player a feeling of freedom is a powerful hybrid combining the advantages of a carefully crafted story and the perceived freedom of choice.
This is made possible by applying indirect control methods to the actions of players:

\begin{enumerate}
    \item \textbf{Constraints:} Restrictions on the number of choices retain the freedom of choice while also preventing cognitive overload through a sheer number of options. There is still uncertainty which option the player chooses but it will ultimately be from the set of pre-defined choices.
    \item \textbf{Goals:} Previously discussed as the central driving force in Section \ref{sec:mechanics-rules} goals are also inherently useful to steer players towards activities connected to these goals. It is thereby only necessary to provide content surrounding these goals. Adding extra content for secondary goals helps to maintain the feeling of freedom while most players will stick to the primary route.
    \item \textbf{Interface:} Both physical and virtual game interfaces are able to exert indirect control on the activities a player might choose. Designing these interfaces with desired activities in mind lowers the probability of players going alternative paths.
    \item \textbf{Visual design:} Drawing attention to different parts of a game's layout allows indirect control on what player's are most likely going to look at. Subsequently, this helps to control where a player might be headed to.
    \item \textbf{Characters:} Non-player characters (NPCs) are especially useful to control the actions of players. Different ways of emotional connections with NPCs decide whether a player obeys, protects, helps or tries to defeat them. It should be noted that this mechanism only works if players empathize with an NPC or have an emotional response.
    \item \textbf{Music:} While music is a great tool to create a desired atmosphere and vibe in games it also acts on the emotional mood of players. The speed, energy and type greatly influences the general feeling of players and can help to steer them along the story.
    \item \textbf{Collusion:} Assigning two goals to NPCs enables them to partake in creating interesting gameplay or stories. Aside from serving their main purposes one of their goals is to assists the designer in ensuring an optimal game experience of the player. This can unravel in NPCs leading players to interesting places or maintaining an optimal tension pattern in the story.
\end{enumerate}